\chapter{Introduction}

\label{intro}

\dropcap{A}{ccurate}, efficient and high-resolution methods of surface water detection are needed for a better water management. Datasets on surface water extent and dynamics are crucial for a better understanding of natural and human-made processes, and as input data for hydrological and hydraulic models. In spite of great progress in the harmonization of freely available satellite data, efficient processing of higher level products remains inherently non-automatable.

Monitoring of surface water extent and dynamics is essential for a better understanding of natural processes, anthropogenic factors as well as climate change. This is becoming more important because water resources are under pressure from economic sectors such as industry, agriculture, energy, tourism, as well as domestic use. Furthermore, water availability is decreasing,  driving more regions into water insecurity \citet{unfao2006watermonitoring, McDonald12042011}.

Recent international agendas on climate change and environment demand objective information on planetary land and water surface conditions and changes, to be able to study the drivers behind them. The \gls{SDG} \citet{un2015resolution} define seventeen challenges to be achieved by 2030, and almost half of them will directly or indirectly require up to date and high-resolution information on surface water. To name some, sustainable management for water and its use for food production, the role of the surface water for the spreading of deceases, and climate change. 

The \gls{IPCC} in the Technical Paper VI on Climate Change and Water defines knowledge gaps in observations and our understanding of the behavior of water. Moreover, with the variety and volumes of \gls{EO} data available today, these gaps frequently mean the lack of proper algorithms to process existing data. The raw data need to be processed properly to derive higher level variables easily interpretable by a broader, frequently non-academic, community. 

\section{Research Questions}

Many attempts have been made during the last decade to establish global scale surface water coverage and occurrence \citet{nasa2003water, lehner2004development, feng2016global, yamazaki2015development}, but most of the studies so far were limited in spatial or temporal resolution, and accuracy. Many questions remain open, and the most important one is the lack of our understanding of how surface water has been changing during the last decades.

Accurate and fully automated detection of surface water from multi-temporal satellite data at high spatial resolution is hard. For efficient and accurate surface water detection from EO data, we need to address issues of global objectivity, accuracy, applicability, as well as provide access for a broad range of users. In the present study, this will be done by answering the following questions:

\begin{itemize}
	\item \textbf{What are the limits of automated surface water detection methods?}
	
	\item \textbf{How to deal with typical issues such as accounting for clouds, hill shadows, snow, ice and mixed urban and rural areas?}
	
	\item \textbf{How to extract the maximum information from very noisy images, where surface water can be only partially visible?}
	
	\item \textbf{How to upscale methods to the global scale?}
	
	\item \textbf{How to streamline the use of satellite images from different passive optical and active radar satellite sensors and what observation frequency can be achieved in this case?}
\end{itemize}

\section{Contributions}

The contributions can be summarized as follows:

\begin{itemize}
	\item A new method ($M_1$) for accurate \textbf{surface water} detection has been developed, based on local thresholding of spectral indices computed from multispectral satellite datasets. The method is demonstrated to perform better than existing methods to discriminate surface water from noisy satellite images. The limitations and the prospects are discussed.
	
	\item A probabilistic method ($M_2$) is developed to \textbf{reconstruct surface water} from satellite images where surface water is only partially observed (due to limited swath, atmospheric noise or snow/ice cover). It is shown that the new method can be used to provide accurate estimates of reservoir surface water area from noisy satellite images.
	
	\item A statistical method ($M_3$) was developed to estimate global-scale \textbf{surface water changes} from medium-resolution multitemporal and multispectral satellite data. The method was applied to process more than a petabyte of Landsat data to identify surface water changes globally.
	
	\item An in-depth \textbf{small reservoir study} was conducted, aiming at the reconstruction of surface water area dynamics from satellite data. Here, we make use of the above two methods ($M_1$ and $M_2$)  to process satellite data from multiple passive multispectral optical and radar sensors (Landsat, ASTER, Sentinel-2 and Sentinel-1). The resulting water masks were validated against in-situ water level observations. A very high correlation was obtained for both cloud-free and noisy images. 
	
	\item To address global water challenges, the first planetary-scale analysis of three decades of satellite images has been performed, quantifying Earth's surface water changes at the 30m spatial resolution and occurring during the last three decades. Two areas were identified that contribute most to surface water gains (Tibetan Plateau) and losses (Aral Sea). The results of the study are made freely available in the form of a website (\url{http://aqua-monitor.deltares.nl}) with the help of the parallel satellite data processing platform Google Earth Engine. The study was performed by applying method $M_3$.
	
	\item Additionally, the same study analyzes surface water changes along the 40km coastal buffer zone globally. The Chinese coast was identified as the largest contributor to coastal changes, when aggregated by country, mainly due to land-reclamation projects.
	
	\item Permanent surface water mask for the Murray-Darling basin has been estimated using the method $M_1$, \gls{HAND}, and supervised classification for topographically difficult areas. The resulting surface water mask has been made available for inspection in the form of a website (\url{http://osm-water.appspot.com}). The water mask developed for Murray-Darling River Basin was compared to the surface water vector dataset extracted from \gls{OSM} and a potential water mask derived from the 30m digital elevation model (\gls{SRTM}). The positional accuracy of the rivers is estimated for three river datasets, and the results are discussed with regard to overall surface water coverage and positional differences.    
\end{itemize}

Eventhough most of the methods developed during this research were applied to process multispectral satellite imagery, some of them are also applicable to process other types of imagery, such as backscatter information generated by \gls{SAR}.

The algorithms presented in this research were successfully applied to develop various surface water datasets and software tools, allowing more accurate detection of surface water and contributing to a better understanding of the Earth's surface water extent and dynamics.

A crucial aspect is also related to simplicity and reproducibility of the methods so they can be easily extended to new datasets and will optimally use existing technical infrastructure. In the present research, all methods are shared with the community with an open-source license, to ensure they can be easily reused and extended.

\section{Outline}

The thesis is organized as follows. Chapter \ref{ch2} reviews the relevant literature and existing methods used to detect surface water from freely-available multispectral satellite sensors. Additionally, it mentions Google Earth Engine - a parallel processing platform used to perform most of the analysis used for this research. The results discussed in the present thesis would be impossible to achieve without the adoption of this platform, which revolutionized satellite data processing, enabling planetary scale analysis for remote sensing researchers around the world. 

\begin{figure}[H]
	\includegraphics[width=1.0\textwidth]{01.1-introduction/figures/visual-map}
	\caption{Thesis visual map}
	\label{fig:outline}
\end{figure}

The next four chapters focus on the development of new methods for automated surface water detection. Chapter \ref{ch3} studies in detail various issues related to surface water detection. Here, the method $M_1$ is introduced, based on the Canny edge detector and Otsu thresholding to allow very accurate detection of surface water. 

We will see how surface water can be detected very accurately even for very noisy images. The method has ben validated using in-situ data available for Prosser Creek Reservoir - a reservoir in California, USA. The method was also extended to process images from multiple passive sensor multispectral satellite missions: Landsat 4, Landsat 5, Landsat 7, Landsat 8, ASTER, and Sentinel-2. It was also applied to process \gls{SAR} imagery from Sentinel-1. We also constructed a simple regression model using the highest quality cloud-free images. The model was used to evaluate performance of the method in Chapter \ref{ch4}. 

Chapter \ref{ch4} focuses on the use of probabilistic methods to further improve surface water area estimates ($M_2$). Here we will see how a surface water mask can be reconstructed even with a small number of noise-free pixels available. This is achieved by a two-step approach, where during the first step, a high-resolution bivariate probability density function is constructed from cloud-free images. Then, a method was developed to infer the final surface water mask from the partially-observed water masks. We also validate the results by comparing it to the model constructed in the previous Chapter, to demonstrate that the method performance is improved.

In Chapter \ref{ch5}, a simple, yet powerful statistical method ($M_3$) will be introduced to analyze long-term surface water changes from a long series of images. We will see how simple surface reflectance percentile composite images can be combined with a linear regression to detect surface water changes and how these surface water changes can be distinguished from other processes, such as cloud cover or snow. The main advantage of this approach is that it allows to significantly speed-up parallel processing while preserving information about surface water changes. The method was then applied to process about 1.5 petabytes of images from medium resolution (<30m) Landsat sensors as shown in Chapter \ref{ch6}. The results of the study are also summarized in the form of surface water changes aggregated per river basin globally, as well as along the coastline, aggregated per country.

Chapter \ref{ch7} demonstrates the use of the methods developed in Chapter \ref{ch3} to derive permanent surface water mask for 2013-2015 the for Murray-Darling River Basin in Australia. Three surface water masks will be compared, derived from three alternative sources: 30m multispectral satellite imagery (Landsat 8), \gls{OSM} and \gls{SRTM}. Here, a stepwise approach to detect permanent surface water from percentile composite images will be applied to analyze images measured by Landsat 8 mission. We will also see how the \gls{HAND} dataset can be used as a topographic mask to detect mountainous areas where commission error of surface water detection is present due to shadowing. The resulting water mask is then refined using a supervised classification method based on \gls{CART}, to perform the final clean-up of the dataset.

Chapter \ref{ch8-conclusion} summarizes the main findings, existing challenges and trends related to methods of surface water detection as well as to the Earth Observation (EO) in general.
